
\documentclass[12pt,french,titlepage]{article}
\usepackage[utf8]{inputenc}
\usepackage{babel}
\usepackage[T1]{fontenc}
\usepackage{mathtools}
\usepackage{amssymb}
\usepackage{amsthm}
\usepackage{amsmath}
\usepackage{hyperref}
\usepackage{graphicx}
\usepackage{float}
\usepackage[dvipsnames]{xcolor}
\definecolor{darkWhite}{rgb}{0.94,0.94,0.94}
\usepackage{tcolorbox,listings}
\lstset{
  aboveskip=3mm,
  belowskip=-2mm,
  backgroundcolor=\color{darkWhite},
  basicstyle=\footnotesize,
  breakatwhitespace=false,
  breaklines=true,
  captionpos=bc,
  commentstyle=\color{ForestGreen},
  deletekeywords={...},
  escapeinside={\%*}{*)},
  extendedchars=true,
  keepspaces=true,
  keywordstyle=\color{blue},
  language=C,
  literate=
  {²}{{\textsuperscript{2}}}1
  {⁴}{{\textsuperscript{4}}}1
  {⁶}{{\textsuperscript{6}}}1
  {⁸}{{\textsuperscript{8}}}1
  {€}{{\euro{}}}1
  {é}{{\'e}}1
  {è}{{\`{e}}}1
  {ê}{{\^{e}}}1
  {ë}{{\¨{e}}}1
  {É}{{\'{E}}}1
  {Ê}{{\^{E}}}1
  {û}{{\^{u}}}1
  {ù}{{\`{u}}}1
  {â}{{\^{a}}}1
  {à}{{\`{a}}}1
  {á}{{\'{a}}}1
  {ã}{{\~{a}}}1
  {Á}{{\'{A}}}1
  {Â}{{\^{A}}}1
  {Ã}{{\~{A}}}1
  {ç}{{\c{c}}}1
  {Ç}{{\c{C}}}1
  {õ}{{\~{o}}}1
  {ó}{{\'{o}}}1
  {ô}{{\^{o}}}1
  {Õ}{{\~{O}}}1
  {Ó}{{\'{O}}}1
  {Ô}{{\^{O}}}1
  {î}{{\^{i}}}1
  {Î}{{\^{I}}}1
  {í}{{\'{i}}}1
  {Í}{{\~{Í}}}1,
  %morekeywords={*,...},
  numbers=left,
  numbersep=10pt,
  numberstyle=\tiny\color{black},
  rulecolor=\color{black},
  showspaces=false,
  showstringspaces=false,
  showtabs=false,
  stepnumber=1,
  stringstyle=\color{gray},
  tabsize=4,
  title=\lstname,
}
\lstdefinestyle{frameStyle}{
    basicstyle=\footnotesize,
    numbers=left,
    numbersep=20pt,
    numberstyle=\tiny\color{black}
}
 
\tcbuselibrary{listings,skins,breakable}
 
\newtcbinputlisting{\cinput}[2][]{
    arc=0mm,
    top=0mm,
    bottom=1mm,
    left=3mm,
    right=0mm,
    width=\textwidth,
    %listing engine=listings,
    listing file={#2},
    listing only,
    listing options={style=frameStyle},
    breakable
}
 

\title{TP Interpolation polynômiale et Approximation.}
\medskip

\author{Salmân Abibou \& Rodrigo Ferreira Rodrigues \\
Université Clermont Auvergne\\}
\vfill

\date{\today}

\begin{document}
	\maketitle


	\tableofcontents
	\newpage
	
	\section{Rappel des méthodes}
	On notera $N$ le nombre de points $(x_i,y_i)$ connues.
	\subsection{Méthodes d'interpolation}
	
	\subsubsection{Lagrange}

	On cherche à exprimer le polynôme $P_{N-1}$ de la forme :
	\begin{equation}
	P_{N-1}(x) = \sum_{i=0}^{(N-1)}y_i\cdot l_i(x) \label{lagrange}
	\end{equation}
	où $l_i(x)$ est la fonction cardinale définie par :
	
	\begin{equation}
	l_i(x) = \prod_{j=0,j\neq i}^{(N-1)}\frac{x - x_j}{x_i - x_j} \label{cardinale}
	\end{equation}
	
	\subsubsection{Neville}
	\label{nev}
	Le but de la méthode est d'obtenir le polynôme $P_{N-1}[x_1,x_2,... ,x_N]$ en fonction des deux polynômes de degré inférieur : $P_{N-2}[x_1,x_2,... ,x_{N-1}]$ et $P_{N-2}[x_2,x_3... ,x_{N}]$. Ces deux polynômes se trouvent chacun grâce à deux autres polynômes de degré inférieur, et ainsi de suite...\\
	Ainsi en partant des polynômes d'interpolation $P_0[x_i]$, on arrive à $P_{N-1}$.\\
	Pour cela, on a les formules :\\
	\begin{align}
		\begin{split}
	&\forall x,P_0[x_i]=y_i,i=1,...,N\\
	&\forall x,P_1[x_i,x_{}i+1](x)=\frac{(x-x_{i+k})P_{0}[x_i](x)+(x_i-x)P_{0}[x_{i+1}](x)}{x_i-x_{i+1}},i=1,...,N\\
	&\forall x,P_k[x_i,...,x_{i+1}](x)=\frac{(x-x_{i+k})P_{k-1}[x_i,...,x_{i+k-1}](x)+(x_i-x)P_{k-1}[x_{i+1},...,x{i+k}](x)}{x_i-x_{i+k}},\\ \label{neville}
	&\forall k=2,...,N-1
		\end{split}
	\end{align}
	
	\subsection{Méthodes d'approximation}
	
	L'approximation consiste à obtenir une fonction $f(x)$ dont la distance moyenne entre les $N$ points connues et la courbe soit minimum.\\
	
	\subsubsection{Droite de régression}
	\label{regression}
	Le but est d'approcher la fonction $f(x)$ grâce à une droite. On cherche ainsi les coefficients $a_0$ et $a_1$ tel que $f(x)=a_0+a_1x$.\\
	A partir de système d'équation :\\\\
	$	
		\begin{bmatrix}
		\sum_{i=0}^{N-1}x_i^0&\sum_{i=0}^{N-1}x_i^1\\
		\sum_{i=0}^{N-1}x_i^1&\sum_{i=0}^{N-1}x_i^2\\
		\end{bmatrix}
	$
	$	
		\begin{bmatrix}
		a_0\\a_1\\
		\end{bmatrix}
	$
	$	
		=
		\begin{bmatrix}
		\sum_{i=0}^{N-1}y_ix_i^0\\
		\sum_{i=0}^{N-1}y_ix_i^1\\
		\end{bmatrix}
	$
	\\\\
	On obtient les coefficients :\\\\
	\begin{align*}
		a_1&=\frac{\overline{yx}-\bar{x}\bar{y}}{\bar{x}^2-(\bar{x})^2}\\\\
		a_0&=\frac{\bar{y}\bar{x}^2-\bar{x}\overline{yx}}{\bar{x}^2-(\bar{x})^2}
	\end{align*}
	\subsubsection{Ajustement puissance du type $y = ax^b$}
	\label{axb}
	Le but de cette méthode est aussi d'obtenir une fonction, approchant au mieux les $N$ points connues.\\
	On cherche alors $f(x)=ax^b$.\\
	
	Pour cela, on manipule cette équation :
	
	\begin{align*}
		y&=ax^b\\
		log (y)&=log(ax^b)\\
		log(y)&=log(a)+b\cdot log(x)\\
	\end{align*}
	On pose :
	
	\begin{align*}
	Y&=log(y_i)\\
	X&=log(x_i)\\
	A&=log(a)\\
	\end{align*}
	
	On obtient alors :
	
	\begin{align*}
	Y&=A+b\cdot X\\
	A&=\overline{Y}-b\cdot \overline{X}\\
	a&=10^A\\
	\end{align*}
	
	avec :
	\begin{equation*}
		b=\frac{\overline{yx}-\bar{x}\bar{y}}{\bar{x}^2-(\bar{x})^2}
	\end{equation*}
	
	
	\section{Explication fonctions du programme}
	
	\subsection{Fonction Cardinale}
	
	\lstinputlisting[language=C, firstline=6, lastline=18, frame=trBL, caption = Fonction auxiliaire cardinale.]{main.c}
	\medskip
	Cette fonction a pour rôle de calculer la fonction cardinale à un indice $i$ et pour une valeur $x$ rentré au préalable par l'utilisateur. (voir section \ref{cardinale}) \\
	Cette fonction est auxiliaire à celle de Lagrange.\\
	\subsection{Fonction Lagrange}
	
	\lstinputlisting[language=C, firstline=20, lastline=26, frame=trBL, caption = Fonction Lagrange.]{main.c}
	\medskip
	Cette fonction permet d'obtenir une solution $y$ selon la variable $x$ rentrée par l'utilisateur et les points données.\\
	Pour cela, la fonction utilise la méthode d'interpolation de Lagrange afin d'obtenir un polynôme $P_{N-1}$. (voir section \ref{lagrange})
	\subsection{Fonction Neville}
	\lstinputlisting[language=C, firstline=29, lastline=37, frame=trBL, caption = Fonction Neville.]{main.c}
	\medskip
	Il s'agit d'une fonction récursive simulant la méthode d'interpolation de Neville. Le cran d'arrêt va être la variable $k$ indiquant le degré du polynôme. Quant elle atteint zéro, on renvoie la valeur $y_i$. Le $y$, solution recherché est ensuite calculé grâce aux formules données. (voir section \ref{nev})
	\subsection{Fonction régression}
	\lstinputlisting[language=C, firstline=39, lastline=58, frame=trBL, caption = Fonction de la droite de regression linéaire.]{main.c}
	
	La fonction \textbf{régression} permet d'obtenir les coefficients $a_0$ et $a_1$ de  la droite de régression. (voir section \ref{regression})
	\subsection{Fonction axb}
	\lstinputlisting[language=C, firstline=60, lastline=80, frame=trBL, caption = Fonction de l'ajustement puissance.]{main.c}
	\medskip
	Cette fonction permet le calcul des coefficients $a$ et $b$ d'une fonction du type $ax^b$. (voir section \ref{axb})
	\section{Analyse jeux d'essais}
	
	\subsection{Densité ($D$) de l'eau en fonction de la température (T)}
		\begin{figure}[H]
		\includegraphics[width=\textwidth]{"11.png"}
		\caption{Interpolation de Lagrange pour le jeu d'essai 1.}
		\end{figure}
		
		\begin{figure}[H]
		\includegraphics[width=\textwidth]{"12.png"}
		\caption{Interpolation de Neville pour le jeu d'essai 1.}
		\end{figure}
		
		\begin{figure}[H]
		\includegraphics[width=\textwidth]{"13.png"}
		\caption{Approximation selon la régression linéaire pour le jeu d'essai 1.}
		\end{figure}

	\subsection{Dépenses mensuelles et revenus}
		\begin{figure}[H]
		\includegraphics[width=\textwidth]{"21.png"}
		\caption{Interpolation de Lagrange pour le jeu d'essai 2.}
		\end{figure}
		
		\begin{figure}[H]
		\includegraphics[width=\textwidth]{"22.png"}
		\caption{Interpolation de Neville pour le jeu d'essai 2.}
		\end{figure}
		
		\begin{figure}[H]
		\includegraphics[width=\textwidth]{"23.png"}
		\caption{Approximation selon la régression linéaire pour le jeu d'essai 2.}
		\end{figure}
		
	\subsection{Série $S$ dûe à Ascombe}
		\begin{figure}[H]
		\includegraphics[width=\textwidth]{"31.png"}
		\caption{Interpolation de Lagrange pour le jeu d'essai 3.}
		\end{figure}
		
		\begin{figure}[H]
		\includegraphics[width=\textwidth]{"32.png"}
		\caption{Interpolation de Neville pour le jeu d'essai 3.}
		\end{figure}
		
		\begin{figure}[H]
		\includegraphics[width=\textwidth]{"33.png"}
		\caption{Approximation selon la régression linéaire pour le jeu d'essai 3.}
		\end{figure}
	
	\subsection{Loi de Pareto}
		\begin{figure}[H]
		\includegraphics[width=\textwidth]{"4.png"}
		\caption{Ajustement puissance pour le jeu d'essai 4.}
		\end{figure}
	
	\section{Conclusion}
	
	
	
\end{document}
