
\documentclass[12pt,french,titlepage]{article}
\usepackage[utf8]{inputenc}
\usepackage{babel}
\usepackage[T1]{fontenc}
\usepackage{mathtools}
\usepackage{amssymb}
\usepackage{amsthm}
\usepackage{amsmath}
\usepackage{hyperref}
\usepackage{graphicx}
\usepackage{float}
\usepackage[dvipsnames]{xcolor}
\definecolor{darkWhite}{rgb}{0.94,0.94,0.94}
\usepackage{tcolorbox,listings}
\lstset{
  aboveskip=3mm,
  belowskip=-2mm,
  backgroundcolor=\color{darkWhite},
  basicstyle=\footnotesize,
  breakatwhitespace=false,
  breaklines=true,
  captionpos=bc,
  commentstyle=\color{ForestGreen},
  deletekeywords={...},
  escapeinside={\%*}{*)},
  extendedchars=true,
  keepspaces=true,
  keywordstyle=\color{blue},
  language=C,
  literate=
  {²}{{\textsuperscript{2}}}1
  {⁴}{{\textsuperscript{4}}}1
  {⁶}{{\textsuperscript{6}}}1
  {⁸}{{\textsuperscript{8}}}1
  {€}{{\euro{}}}1
  {é}{{\'e}}1
  {è}{{\`{e}}}1
  {ê}{{\^{e}}}1
  {ë}{{\¨{e}}}1
  {É}{{\'{E}}}1
  {Ê}{{\^{E}}}1
  {û}{{\^{u}}}1
  {ù}{{\`{u}}}1
  {â}{{\^{a}}}1
  {à}{{\`{a}}}1
  {á}{{\'{a}}}1
  {ã}{{\~{a}}}1
  {Á}{{\'{A}}}1
  {Â}{{\^{A}}}1
  {Ã}{{\~{A}}}1
  {ç}{{\c{c}}}1
  {Ç}{{\c{C}}}1
  {õ}{{\~{o}}}1
  {ó}{{\'{o}}}1
  {ô}{{\^{o}}}1
  {Õ}{{\~{O}}}1
  {Ó}{{\'{O}}}1
  {Ô}{{\^{O}}}1
  {î}{{\^{i}}}1
  {Î}{{\^{I}}}1
  {í}{{\'{i}}}1
  {Í}{{\~{Í}}}1,
  %morekeywords={*,...},
  numbers=left,
  numbersep=10pt,
  numberstyle=\tiny\color{black},
  rulecolor=\color{black},
  showspaces=false,
  showstringspaces=false,
  showtabs=false,
  stepnumber=1,
  stringstyle=\color{gray},
  tabsize=4,
  title=\lstname,
}
\lstdefinestyle{frameStyle}{
    basicstyle=\footnotesize,
    numbers=left,
    numbersep=20pt,
    numberstyle=\tiny\color{black}
}
 
\tcbuselibrary{listings,skins,breakable}
 
\newtcbinputlisting{\cinput}[2][]{
    arc=0mm,
    top=0mm,
    bottom=1mm,
    left=3mm,
    right=0mm,
    width=\textwidth,
    %listing engine=listings,
    listing file={#2},
    listing only,
    listing options={style=frameStyle},
    breakable
}
 

\title{TP Interpolation polynômiale et Approximation.}
\medskip

\author{Salmân Abibou \& Rodrigo Ferreira Rodrigues \\
Université Clermont Auvergne\\}
\vfill

\date{\today}

\begin{document}
	\maketitle


	\tableofcontents
	\newpage
	
	\section{Rappel des méthodes}
	On notera $N$ le nombre de points $(x_i,y_i)$ connues.
	\subsection{Méthodes d'interpolation}
	
	\subsubsection{Lagrange}

	On cherche à exprimer le polynôme $P_{N-1}$ de la forme :
	\begin{equation}
	P_{N-1}(x) = \sum_{i=0}^{(N-1)}y_i\cdot l_i(x)
	\end{equation}
	où $l_i(x)$ est la fonction cardinale définie par :
	
	\begin{equation}
	l_i(x) = \prod_{j=0,j\neq i}^{(N-1)}\frac{x - x_j}{x_i - x_j}
	\end{equation}
	
	\subsubsection{Neville}
	
	Le but de la méthode est d'obtenir le polynôme $P_{N-1}[x_1,x_2,... ,x_N]$ en fonction des deux polynômes de degré inférieur : $P_{N-2}[x_1,x_2,... ,x_{N-1}]$ et $P_{N-2}[x_2,x_3... ,x_{N}]$. Ces deux polynômes se trouvent chacun grâce à deux autres polynômes de degré inférieur, et ainsi de suite...\\
	Ainsi en partant des polynômes d'interpolation $P_0[x_i]$, on arrive à $P_{N-1}$.\\
	Pour cela, on a les formules :\\
	\begin{align}
		\begin{split}
	&\forall x,P_0[x_i]=y_i,i=1,...,N\\
	&\forall x,P_1[x_i,x_{}i+1](x)=\frac{(x-x_{i+k})P_{0}[x_i](x)+(x_i-x)P_{0}[x_{i+1}](x)}{x_i-x_{i+1}},i=1,...,N\\
	&\forall x,P_k[x_i,...,x_{i+1}](x)=\frac{(x-x_{i+k})P_{k-1}[x_i,...,x_{i+k-1}](x)+(x_i-x)P_{k-1}[x_{i+1},...,x{i+k}(x)]}{x_i-x_{i+k}},\\
	&\forall k=2,...,N-1
		\end{split}
	\end{align}
	
	\subsection{Méthodes d'approximation}
	
	\subsubsection{Droite de régression}
	
	\subsubsection{Ajustement puissance du type $y = ax^b$}
	
	\section{Explication fonctions du porgramme}
	
	\subsection{Fonction Cardinale}
	
	\subsection{Fonction Lagrange}
	
	\subsection{Fonction Neville}
	
	\subsection{Fonction régression}
	
	\subsection{Fonction axb}
	
	\section{Analyse jeux d'essais}
	
	\subsection{Densité ($D$) de l'eau en fonction de la température (T)}
	
	\subsection{Dépenses mensuelles et revenus}
	
	\subsection{Série $S$ dûe à Ascombe}
	
	\subsection{Loi de Pareto}
	
	\section{Conclusion}
	
	
	
\end{document}
