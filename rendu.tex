
\documentclass[12pt,french,titlepage]{article}
\usepackage[utf8]{inputenc}
\usepackage{babel}
\usepackage[T1]{fontenc}
\usepackage{mathtools}
\usepackage{amssymb}
\usepackage{amsthm}
\usepackage{amsmath}
\usepackage{hyperref}
\usepackage{graphicx}
\usepackage{float}
\usepackage[dvipsnames]{xcolor}
\definecolor{darkWhite}{rgb}{0.94,0.94,0.94}
\usepackage{tcolorbox,listings}
\lstset{
  aboveskip=3mm,
  belowskip=-2mm,
  backgroundcolor=\color{darkWhite},
  basicstyle=\footnotesize,
  breakatwhitespace=false,
  breaklines=true,
  captionpos=bc,
  commentstyle=\color{ForestGreen},
  deletekeywords={...},
  escapeinside={\%*}{*)},
  extendedchars=true,
  keepspaces=true,
  keywordstyle=\color{blue},
  language=C,
  literate=
  {²}{{\textsuperscript{2}}}1
  {⁴}{{\textsuperscript{4}}}1
  {⁶}{{\textsuperscript{6}}}1
  {⁸}{{\textsuperscript{8}}}1
  {€}{{\euro{}}}1
  {é}{{\'e}}1
  {è}{{\`{e}}}1
  {ê}{{\^{e}}}1
  {ë}{{\¨{e}}}1
  {É}{{\'{E}}}1
  {Ê}{{\^{E}}}1
  {û}{{\^{u}}}1
  {ù}{{\`{u}}}1
  {â}{{\^{a}}}1
  {à}{{\`{a}}}1
  {á}{{\'{a}}}1
  {ã}{{\~{a}}}1
  {Á}{{\'{A}}}1
  {Â}{{\^{A}}}1
  {Ã}{{\~{A}}}1
  {ç}{{\c{c}}}1
  {Ç}{{\c{C}}}1
  {õ}{{\~{o}}}1
  {ó}{{\'{o}}}1
  {ô}{{\^{o}}}1
  {Õ}{{\~{O}}}1
  {Ó}{{\'{O}}}1
  {Ô}{{\^{O}}}1
  {î}{{\^{i}}}1
  {Î}{{\^{I}}}1
  {í}{{\'{i}}}1
  {Í}{{\~{Í}}}1,
  %morekeywords={*,...},
  numbers=left,
  numbersep=10pt,
  numberstyle=\tiny\color{black},
  rulecolor=\color{black},
  showspaces=false,
  showstringspaces=false,
  showtabs=false,
  stepnumber=1,
  stringstyle=\color{gray},
  tabsize=4,
  title=\lstname,
}
\lstdefinestyle{frameStyle}{
    basicstyle=\footnotesize,
    numbers=left,
    numbersep=20pt,
    numberstyle=\tiny\color{black}
}
 
\tcbuselibrary{listings,skins,breakable}
 
\newtcbinputlisting{\cinput}[2][]{
    arc=0mm,
    top=0mm,
    bottom=1mm,
    left=3mm,
    right=0mm,
    width=\textwidth,
    %listing engine=listings,
    listing file={#2},
    listing only,
    listing options={style=frameStyle},
    breakable
}
 

\title{TP Interpolation polynômiale et Approximation.}
\medskip

\author{Salmân Abibou \& Rodrigo Ferreira Rodrigues \\
Université Clermont Auvergne\\}
\vfill

\date{\today}

\begin{document}
	\maketitle


	\tableofcontents
	\newpage
	
	\section{Rappel des méthodes}
	On notera $N$ le nombre de points $(x_i,y_i)$ connues.
	\subsection{Méthodes d'interpolation}
	
	\subsubsection{Lagrange}

	On cherche à exprimer le polynôme $P_{N-1}$ de la forme :
	\begin{equation}
	P_{N-1}(x) = \sum_{i=0}^{(N-1)}y_i\cdot l_i(x) \label{lagrange}
	\end{equation}
	où $l_i(x)$ est la fonction cardinale définie par :
	
	\begin{equation}
	l_i(x) = \prod_{j=0,j\neq i}^{(N-1)}\frac{x - x_j}{x_i - x_j} \label{cardinale}
	\end{equation}
	
	\subsubsection{Neville}
	\label{nev}
	Le but de la méthode est d'obtenir le polynôme $P_{N-1}[x_1,x_2,... ,x_N]$ en fonction des deux polynômes de degré inférieur : $P_{N-2}[x_1,x_2,... ,x_{N-1}]$ et $P_{N-2}[x_2,x_3... ,x_{N}]$. Ces deux polynômes se trouvent chacun grâce à deux autres polynômes de degré inférieur, et ainsi de suite...\\
	Ainsi en partant des polynômes d'interpolation $P_0[x_i]$, on arrive à $P_{N-1}$.\\
	Pour cela, on a les formules :\\
	\begin{align}
		\begin{split}
	&\forall x,P_0[x_i]=y_i,i=1,...,N\\
	&\forall x,P_1[x_i,x_{}i+1](x)=\frac{(x-x_{i+k})P_{0}[x_i](x)+(x_i-x)P_{0}[x_{i+1}](x)}{x_i-x_{i+1}},i=1,...,N\\
	&\forall x,P_k[x_i,...,x_{i+1}](x)=\frac{(x-x_{i+k})P_{k-1}[x_i,...,x_{i+k-1}](x)+(x_i-x)P_{k-1}[x_{i+1},...,x{i+k}](x)}{x_i-x_{i+k}},\\ \label{neville}
	&\forall k=2,...,N-1
		\end{split}
	\end{align}
	
	\subsection{Méthodes d'approximation}
	
	L'approximation consiste à obtenir une fonction $f(x)$ dont la distance moyenne entre les $N$ points connues et la courbe soit minimum.\\
	
	\subsubsection{Droite de régression}
	\label{regression}
	Le but est d'approcher la fonction $f(x)$ grâce à une droite. On cherche ainsi les coefficients $a_0$ et $a_1$ tel que $f(x)=a_0+a_1x$.\\
	A partir de système d'équation :\\\\
	$	
		\begin{bmatrix}
		\sum_{i=0}^{N-1}x_i^0&\sum_{i=0}^{N-1}x_i^1\\
		\sum_{i=0}^{N-1}x_i^1&\sum_{i=0}^{N-1}x_i^2\\
		\end{bmatrix}
	$
	$	
		\begin{bmatrix}
		a_0\\a_1\\
		\end{bmatrix}
	$
	$	
		=
		\begin{bmatrix}
		\sum_{i=0}^{N-1}y_ix_i^0\\
		\sum_{i=0}^{N-1}y_ix_i^1\\
		\end{bmatrix}
	$
	\\\\
	On obtient les coefficients :\\\\
	\begin{align*}
		a_1&=\frac{\overline{yx}-\bar{x}\bar{y}}{\bar{x}^2-(\bar{x})^2}\\\\
		a_0&=\frac{\bar{y}\bar{x}^2-\bar{x}\overline{yx}}{\bar{x}^2-(\bar{x})^2}
	\end{align*}
	\subsubsection{Ajustement puissance du type $y = ax^b$}
	\label{axb}
	Le but de cette méthode est aussi d'obtenir une fonction, approchant au mieux les $N$ points connues.\\
	On cherche alors $f(x)=ax^b$.\\
	
	Pour cela, on manipule cette équation :
	
	\begin{align*}
		y&=ax^b\\
		log (y)&=log(ax^b)\\
		log(y)&=log(a)+b\cdot log(x)\\
	\end{align*}
	On pose :
	
	\begin{align*}
	Y&=log(y_i)\\
	X&=log(x_i)\\
	A&=log(a)\\
	\end{align*}
	
	On obtient alors :
	
	\begin{align*}
	Y&=A+b\cdot X\\
	A&=\overline{Y}-b\cdot \overline{X}\\
	a&=10^A\\
	\end{align*}
	
	avec :
	\begin{equation*}
		b=\frac{\overline{yx}-\bar{x}\bar{y}}{\bar{x}^2-(\bar{x})^2}
	\end{equation*}
	
	
	\section{Explication fonctions du programme}
	
	\subsection{Fonction Cardinale}
	
	\lstinputlisting[language=C, firstline=6, lastline=18, frame=trBL, caption = Fonction auxiliaire cardinale.]{main.c}
	\medskip
	Cette fonction a pour rôle de calculer la fonction cardinale à un indice $i$ et pour une valeur $x$ rentré au préalable par l'utilisateur. (voir section \ref{cardinale}) \\
	Cette fonction est auxiliaire à celle de Lagrange.\\\\
	\textbf{Entrée} : N, le nombre de points, le réel x, un indice i et un tableau contenant les valeurs x du jeu\\
	\textbf{Sortie} : $l_i(x)$\\
	\textbf{Complexité spatiale} : $O(n)$\\
	\textbf{Complexité temporelle} : $O(n)$\\
	\subsection{Fonction Lagrange}
	
	\lstinputlisting[language=C, firstline=20, lastline=26, frame=trBL, caption = Fonction Lagrange.]{main.c}
	\medskip
	Cette fonction permet d'obtenir une solution $y$ selon la variable $x$ rentrée par l'utilisateur et les points données.\\
	Pour cela, la fonction utilise la méthode d'interpolation de Lagrange afin d'obtenir un polynôme $P_{N-1}$. (voir section \ref{lagrange})\\
	
	\textbf{Entrée} : N, le nombre de points, le réel x et un tableau contenant les valeurs x et y du jeu\\
	\textbf{Sortie} : $y$\\
	\textbf{Complexité spatiale} : $O(n^2)$\\
	\textbf{Complexité temporelle} :$O(n^2)$\\
	
	\subsection{Fonction Neville}
	\lstinputlisting[language=C, firstline=29, lastline=37, frame=trBL, caption = Fonction Neville.]{main.c}
	\medskip
	Il s'agit d'une fonction récursive simulant la méthode d'interpolation de Neville. Le cran d'arrêt va être la variable $k$ indiquant le degré du polynôme. Quant elle atteint zéro, on renvoie la valeur $y_i$. Le $y$, solution recherché est ensuite calculé grâce aux formules données. (voir section \ref{nev})\\
	
	\textbf{Entrée} : N, le nombre de points, le réel x, un indice i et un tableau contenant les valeurs x et y du jeu\\
	\textbf{Sortie} : $y$\\
	\textbf{Complexité spatiale} : $O(1)$\\
	\textbf{Complexité temporelle} : $O(log(n)) $\\
	
	\subsection{Fonction régression}
	\lstinputlisting[language=C, firstline=39, lastline=58, frame=trBL, caption = Fonction de la droite de regression linéaire.]{main.c}
	
	La fonction \textbf{régression} permet d'obtenir les coefficients $a_0$ et $a_1$ de  la droite de régression. (voir section \ref{regression})\\
	
	\textbf{Entrée} : N, le nombre de points et un tableau contenant les valeurs x et y du jeu\\
	\textbf{Sortie} : $a_1$ et $a_0$, les coefficients de la fonction recherchée, stockées dans un tableau\\
	\textbf{Complexité spatiale} : $O(n)$\\
	\textbf{Complexité temporelle} : $O(n) $\\
	
	\subsection{Fonction axb}
	\lstinputlisting[language=C, firstline=60, lastline=80, frame=trBL, caption = Fonction de l'ajustement puissance.]{main.c}
	\medskip
	Cette fonction permet le calcul des coefficients $a$ et $b$ d'une fonction du type $ax^b$. (voir section \ref{axb})\\
	
	\textbf{Entrée} : N, le nombre de points et un tableau contenant les valeurs x et y du jeu\\
	\textbf{Sortie} : $a$ et $b$, les coefficients de la fonction recherchée, stockées dans un tableau\\
	\textbf{Complexité spatiale} : $O(n)$\\
	\textbf{Complexité temporelle} : $O(n) $\\
	
	\section{Analyse jeux d'essais}
	
	\subsection{Densité ($D$) de l'eau en fonction de la température (T)}
		Les données connues représentent la densité de l'eau (en $t/m^3$) en fonction de la température (en $°C$).\\
		On remarque avec ces données que la densité diminue alors que la température augmente.\\
		\begin{figure}[H]
		\includegraphics[width=\textwidth]{"11.png"}
		\caption{Interpolation de Lagrange pour le jeu d'essai 1.}
		\end{figure}
		
		\begin{figure}[H]
		\includegraphics[width=\textwidth]{"12.png"}
		\caption{Interpolation de Neville pour le jeu d'essai 1.}
		\end{figure}
	
		\begin{itemize}
			\item On remarque qu'on obtient les mêmes graphiques $\rightarrow$ unicité des polynômes
			\item On remarque qu'à partir de $x=36$ le polynôme croit de façon exponentielle $\rightarrow$ le polynôme n'est plus fiable
		\end{itemize}
		
		\begin{figure}[H]
		\includegraphics[width=\textwidth]{"13.png"}
		\caption{Approximation selon la régression linéaire pour le jeu d'essai 1.}
		\end{figure}
	
		\begin{itemize}
			\item la fonction décroît $\rightarrow$ représente bien la situation
		\end{itemize}

	\subsection{Dépenses mensuelles et revenus}
		\begin{figure}[H]
		\includegraphics[width=\textwidth]{"21.png"}
		\caption{Interpolation de Lagrange pour le jeu d'essai 2.}
		\end{figure}
		
		\begin{figure}[H]
		\includegraphics[width=\textwidth]{"22.png"}
		\caption{Interpolation de Neville pour le jeu d'essai 2.}
		\end{figure}
		
		\begin{figure}[H]
		\includegraphics[width=\textwidth]{"23.png"}
		\caption{Approximation selon la régression linéaire pour le jeu d'essai 2.}
		\end{figure}
		
	\subsection{Série $S$ dûe à Ascombe}
	
	Cette série fait partie du quartet d'Anscombe qui est constitué de quatre ensembles de données qui ont les mêmes propriétés statistiques simples. Ils ont été construits par le statisticien Francis Anscombe dans le but de démontrer l'importance de tracer les graphiques avant d'analyser des données, car cela permet notamment d'estimer l'incidence des données aberrantes sur les différentes indices statistiques que l'on pourrait calculer\footnote{\url{https://fr.wikipedia.org/wiki/Quartet_d\%27Anscombe}}.\\
	Cette série contient $11$ points de données (représentés en point rouges) avec une moyenne des abcisses de $9$ et une moyenne des ordonnées de $7.5$.
		\begin{figure}[H]
		
Ici on a déterminé les $y_i$ de chacun des $x_i$ allant de $0$ à $20$ par les polynômes obtenus respectivement par la méthode de Lagrange et de Neville et on les a représentés sur les graphiques ci-dessous.
		\includegraphics[width=\textwidth]{"31.png"}
		\caption{Interpolation de Lagrange pour le jeu d'essai 3.}
		\end{figure}
		
		\begin{figure}[H]
		\includegraphics[width=\textwidth]{"32.png"}
		\caption{Interpolation de Neville pour le jeu d'essai 3.}
		\end{figure}
		
Les polynômes d'interpolation de Lagrange et de Neville n'approchent pas bien l'ensemble de données là où les données ne sont pas connues.\\
En fait, lorsqu'on augmente le nombre de points, on constate que le polynôme se met à osciller fortement entre les points $x_i$ avec une amplitude de plus en plus grande, comme l'illustre les figures ci-dessus.
		\begin{figure}[H]
		\includegraphics[width=\textwidth]{"33.png"}
		\caption{Approximation selon la régression linéaire pour le jeu d'essai 3.}
		\end{figure}
Le but de l'approximation par la droite de régression est de trouver la droite qui passe par le plus de points de l'ensemble de données.
	\subsection{Loi de Pareto}
On vérifie la loi de Pareto entre un revenu $x$ et le nombre de personnes $y$ ayant un revenu supérieur à $x$. On remarque après calcul qu'un pourcentage cumulé de $82.08\%$ ont un revenu supérieur à $50$.
		\begin{figure}[H]
		\includegraphics[width=\textwidth]{"4.png"}
		\caption{Ajustement puissance pour le jeu d'essai 4.}
		\end{figure}
Plus le revenu $x$ augmente, plus vite le nombre de personnes $y$ ayant un revenu supérieur à $x$ diminue de façon exponentielle et tend vers $0$.¨
	\section{Conclusion}
	
Pour approcher une fonction avec des polynômes, on peut préférer utiliser des polynômes par morceaux. Dans ce cas, pour améliorer l'approximation, on augmente le nombre de morceaux et non le degré des polynômes. (Segmentation)	
	
\end{document}
